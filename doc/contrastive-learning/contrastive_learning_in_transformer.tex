\documentclass[a4paper]{article}

\usepackage[pages=all, color=black, position={current page.south}, placement=bottom, scale=1, opacity=1, vshift=5mm]{background}

\usepackage[margin=1in]{geometry} % full-width

% AMS Packages
\usepackage{amsmath}
\usepackage{amsthm}
\usepackage{amssymb}

% Unicode
\usepackage[utf8]{inputenc}
\usepackage{hyperref}
\hypersetup{
	unicode,
%	colorlinks,
%	breaklinks,
%	urlcolor=cyan, 
%	linkcolor=blue, 
	pdfauthor={Wenfeng Feng, Xin Zhang},
	pdftitle={Learning by Predicting future},
	pdfsubject={Learning by Predicting future},
	pdfkeywords={Contrastive Learning, latent representation learning},
	pdfproducer={LaTeX},
	pdfcreator={pdflatex}
}

% Vietnamese
%\usepackage{vntex}

% Natbib
\usepackage[sort&compress,numbers,square]{natbib}
\bibliographystyle{mplainnat}

% Theorem, Lemma, etc
\theoremstyle{plain}
\newtheorem{theorem}{Theorem}
\newtheorem{corollary}[theorem]{Corollary}
\newtheorem{lemma}[theorem]{Lemma}
\newtheorem{claim}{Claim}[theorem]
\newtheorem{axiom}[theorem]{Axiom}
\newtheorem{conjecture}[theorem]{Conjecture}
\newtheorem{fact}[theorem]{Fact}
\newtheorem{hypothesis}[theorem]{Hypothesis}
\newtheorem{assumption}[theorem]{Assumption}
\newtheorem{proposition}[theorem]{Proposition}
\newtheorem{criterion}[theorem]{Criterion}
\theoremstyle{definition}
\newtheorem{definition}[theorem]{Definition}
\newtheorem{example}[theorem]{Example}
\newtheorem{remark}[theorem]{Remark}
\newtheorem{problem}[theorem]{Problem}
\newtheorem{principle}[theorem]{Principle}

\usepackage{graphicx, color}
\graphicspath{{fig/}}

%\usepackage[linesnumbered,ruled,vlined,commentsnumbered]{algorithm2e} % use algorithm2e for typesetting algorithms
\usepackage{algorithm, algpseudocode} % use algorithm and algorithmicx for typesetting algorithms
\usepackage{mathrsfs} % for \mathscr command

\usepackage{lipsum}

% Author info
\title{
	Latent representation learning by predicting near future
	}
\author{Wenfeng Feng$^1$ \and Xin Zhang$^1$}

\date{
	% $^1$Organization 1 \\ \texttt{\{auth1, auth3\}@org1.edu}\\%
	% $^2$Organization 2 \\ \texttt{auth3@inst2.edu}\\[2ex]%
%	\today
}

\begin{document}
	\maketitle
	
	\begin{abstract}
		Learning representation directly in the latent space has make progress recently.
		
		\noindent\textbf{Keywords:} , , 
	\end{abstract}

	\tableofcontents
	
	\section{Introduction}
	\label{sec:intro}
	
	We do not predict future observations $x_{t+k}$ directly with a generative model $p_k(x_{t+k}|x_{\le t})$.
	Instead we model a density ratio which preserves the mutual information between $x_{t+k}$ and $x_{\le t}$.
	\cite{van2018representation}
	
	\subsection{Loss function}
	\label{subsec:loss}

	\paragraph*{Barlow Twins} \cite{zbontar2021barlow}
	measuring the cross-correlation matrix between the outputs of two identical networks fed with distorted versions of a sample,
	and making it as close to the identity matrix as possible.

	
	\subsection{How to avoid collapse}
	\label{subsec:avoid-collapse}

	\paragraph*{BN}
	Batch normal is a method of avoiding collapse.

	\begin{table}[ht]
		\centering
		\begin{tabular}{|c|c|}
			\hline
			\textbf{Odd} & \textbf{Even} \\
			\hline
			One & Two \\
			\hline
			Three & Four \\
			\hline
		\end{tabular}
		\caption{This is a table}
		\label{tbl:1}
	\end{table}

	Table~\ref*{tbl:1} is an example of a table.
	
	\section{Related works}
	\label{sec:related}
	
	\subsection{How to generate positive sample pairs}

	\paragraph*{Dropout} \cite{gao2021simcse} In sentence similarity, for creating positive sample pairs, the same input sentence e.g. 'Two dogs are running' is passed into the pre-trained encoder twice and two different embeddings are obtained by applying independently sampled dropout masks.  

	\section{Method}

	\subsection{Motivation and Intuitions}
	Please see Figure~\ref{fig:example} for the visualization of our method.
	\begin{figure}[ht]
		\centering
		\includegraphics[width=0.3\textwidth]{example}
		\caption{An example of a figure}
		\label{fig:example}
	\end{figure}

	\cite{van2018representation}p2-3
	When predicting future information using auto-regressive model $p(x_t|x_{<t})$, we instead encode the target $x_t$ and context $x_{<t}$ into a compact distributed vector representations $z_t$ and $z_{<t}$.
	We learn the underlying shared information between the target and context by maximizing the mutual information of the latent representation between them $z_t$ and $z_{<t}$ defined as
	\begin{equation}
		I(z_t; z_{<t}) = 
	\end{equation}

%	\newpage
	\bibliography{refs}
		
\end{document}