\documentclass[11pt,a4paper,sans]{moderncv}        % possible options include font size ('10pt', '11pt' and '12pt'), paper size ('a4paper', 'letterpaper', 'a5paper', 'legalpaper', 'executivepaper' and 'landscape') and font family ('sans' and 'roman')
\usepackage{multicol}
% moderncv themes
\moderncvstyle{classic}                             % style options are 'casual' (default), 'classic', 'oldstyle' and 'banking'
\moderncvcolor{blue}                               % color options 'blue' (default), 'orange', 'green', 'red', 'purple', 'grey' and 'black'
%\renewcommand{\familydefault}{\sfdefault}         % to set the default font; use '\sfdefault' for the default sans serif font, '\rmdefault' for the default roman one, or any tex font name
%\nopagenumbers{}                                  % uncomment to suppress automatic page numbering for CVs longer than one page

% character encoding
%\usepackage[utf8]{inputenc}                       % if you are not using xelatex ou lualatex, replace by the encoding you are using
%\usepackage{CJKutf8}                              % if you need to use CJK to typeset your resume in Chinese, Japanese or Korean

% adjust the page margins
\usepackage[scale=0.81]{geometry}
%\setlength{\hintscolumnwidth}{3cm}                % if you want to change the width of the column with the dates
%\setlength{\makecvtitlenamewidth}{10cm}           % for the 'classic' style, if you want to force the width allocated to your name and avoid line breaks. be careful though, the length is normally calculated to avoid any overlap with your personal info; use this at your own typographical risks...

% personal data
\name{Wenfeng}{Feng}
\title{ }                               % optional, remove / comment the line if not wanted
\address{School of Computer Science and Technology}{Henan Polytechnic University, China}%{UK}
% \address{Kawazu 680-4}{Iizuka-shi Fukuoka 8208502}{Japan}% optional, remove / comment the line if not wanted; the "postcode city" and "country" arguments can be omitted or provided empty
\email{fengwenfeng@gmail.com}                               % optional, remove / comment the line if not wanted
% \homepage{github.com/keepsimpler}                         % optional, remove / comment the line if not wanted
%\social[linkedin]{oliver.feng}                        % optional, remove / comment the line if not wanted
%\social[twitter]{showu}                             % optional, remove / comment the line if not wanted
\social[github]{github.com/keepsimpler}                              % optional, remove / comment the line if not wanted
%\extrainfo{additional information}                 % optional, remove / comment the line if not wanted
%\photo[64pt][0.4pt]{picture}                       % optional, remove / comment the line if not wanted; '64pt' is the height the picture must be resized to, 0.4pt is the thickness of the frame around it (put it to 0pt for no frame) and 'picture' is the name of the picture file
%\quote{Some quote}                                 % optional, remove / comment the line if not wanted
\phone[mobile]{+86~139~3911~9371}                   % optional, remove / comment the line if not wanted; the optional "type" of the phone can be "mobile" (default), "fixed" or "fax"
%\phone[mobile]{+44~(0794)~842~6379}
%\phone[fax]{+3~(456)~789~012}

% to show numerical labels in the bibliography (default is to show no labels); only useful if you make citations in your resume
%\makeatletter
%\renewcommand*{\bibliographyitemlabel}{\@biblabel{\arabic{enumiv}}}
%\makeatother
%\renewcommand*{\bibliographyitemlabel}{[\arabic{enumiv}]}% CONSIDER REPLACING THE ABOVE BY THIS

% bibliography with mutiple entries
%\usepackage{multibib}
%\newcites{book,misc}{{Books},{Others}}
%----------------------------------------------------------------------------------
%            content
%----------------------------------------------------------------------------------
\begin{document}
%\begin{CJK*}{UTF8}{gbsn}                          % to typeset your resume in Chinese using CJK
%-----       resume       ---------------------------------------------------------
\makecvtitle

\section{Experience}

\cventry{2016--Present}{Associate Professor}{}{School of Computer Sci. and Tech.}{Henan Polytechnic University}{
\begin{itemize}
  % \item Provide a wide range of computing support, resources and services for the HPU campus, including networks, cloud computing platform, business systems, information security, etc.
  \item Study on new deep neural network architectures applying into computer vision and natural language processing. 
 \end{itemize}  
}
\cventry{2015--2016}{Postdoctoral Researcher}{}{School of Geography and the Environment}{University of Oxford}{
 \begin{itemize}
  \item Developed a theoretical framework to explore the built-in mechanisms of critical transitions and stability for mutualistic ecological systems.
  \item Established a mathematical mean-field mutualistic model that captures essential features of mutualistic systems besides structural features like degree heterogeneity and modularity.
  \item Mathematically explored alternative stable states, cusp catastrophe, hysteresis, etc. of ecological systems.
  \item Whether `dot' eigenvalue or `semicircle' eigenvalue of the community matrix dominate determine two kinds of critical transitions, one kind leads to extinction of all species, the other kind leads to splitting of species densities.
  \item The splitting critical transitions is more difficult to anticipate then the consistent critical transitions.
  % \item Degree heterogeneity of species both increase and decrease the resilience, persistence and the total abundances of mutualistic systems at specific conditions.
  \item Developed a R package \textit{StabEco} exploring the deterministic and stochastic dynamics of ecological systems.
 \end{itemize}  
}
\cventry{2013--2015}{Postdoctoral Researcher}{}{Dept. of Biosci. and Bioinfo.}{Kyushu Institute of Technology}{
  \begin{itemize}
  \item Proposed a theoretical method for estimating the dominant eigenvalue of quantitative bipartite networks by extending spectral graph theory.
  \item Provided a theoretical prediction that the heterogeneity of node degrees and link weights primarily determines the local stability of mutualistic ecological communities.
  %\item Proved that variances of the aggregate biomass in competitive and mutualistic communities depend on species richness and identical intrinsic growth rates under symmetric interactions.
  %\item Proved that sum of the variances of the biomasses of constitute species depend on distribution of eigenvalues of the community matrix which further depend on structure of ecological networks.
 \end{itemize}
 }

\cventry{2006--2013}{Assistant Professor}{}{Dept. of Computer Sci. and Tech.}{Henan Polytechnic University}{
\begin{itemize}
\item Teaching courses: Computer Network Architecture, Network Programming, Distributed Systems
\item Developed a SOFM neural network algorithm to estimate construction cost.
%\item Manager of the campus network
\end{itemize}} 

% \cventry{2006--2008}{Postdoctoral Researcher}{}{Dept. of Electronic Engineering}{Tsinghua University}{   % arguments 3 to 6 can be left empty
% \begin{itemize}%
% \item Developed a peer-to-peer(P2P) traffic identification method using machine learning algorithm.
% \item Implemented a SIP-based P2P video streaming software (Copyright No. 2008SRBJ3221).
% \item Developed a traffic flow prediction algorithm based on the traffic network data set of Beijing city.
% \end{itemize}}

% \cventry{1996--2000}{Software Engineer}{}{Henan Branch of China Construction Bank}{}{
%   \begin{itemize}
%   \item Worked on maintenance and development of the core operating system of bank. 
%   \end{itemize}
% }

\section{Education}
\cventry{2003--2006}{Ph.D.}{}{in Computer Science}{Beijing Institute of Technology, China}{Thesis: Sketch Data Structures and Algorithms and their Applications in Real-time Network Data Stream Mining}  % arguments 3 to 6 can be left empty
\cventry{2000--2003}{M.S.}{}{in Computer Software and Theory}{Beijing Institute of Technology, China}{}
% \cventry{1992--1996}{B.E.}{}{in Computer Applied Technology}{North China University of Technology, China}{}

%\clearpage

\section{Funding}
\cvitem{2012--2014}{Program for New Century Excellent Talents in University of China (no. NCET-11-0942) \newline Stability of theoretical and empirical complex networks}
\cvitem{2008--2010}{Program of National Natural Science Foundation of China (no. 60703053) \newline Modelling of dynamic heterogeneity of self-organized overlay network and its optimization}

% \section{Projects}
% \cvitem{2003--2004}{Decision support system for freight marketing, Ministry of Railways, China. Tech. SAS Base/DW/DM.}
% \cvitem{2002--2003}{Customer analysis system, Shanxi branch of China Construction Bank. Tech. DataStage ETL / RedBrick DW / SQL / JSP}
% \cvitem{2004--2005}{nTracker network flow management system. Tech. Real time data streaming.}



% Publications from a BibTeX file without multibib
%  for numerical labels: \renewcommand{\bibliographyitemlabel}{\@biblabel{\arabic{enumiv}}}% CONSIDER MERGING WITH PREAMBLE PART
%  to redefine the heading string ("Publications"): \renewcommand{\refname}{Articles}
\renewcommand{\bibliographyitemlabel}{{\arabic{enumiv}}}
%\renewcommand{\refname}{Selected Publications}
\nocite{*}
\begin{thebibliography}{10}

\bibitem{feng2021replicator}
\textbf{Wenfeng Feng}.
\newblock Replicator: a new deep learning architecture.
\newblock {\em in Review}.

\bibitem{feng2021replicator}
\textbf{Wenfeng Feng} and Xin Zhang.
\newblock ResNetX: a more disordered and deeper network architecture.
\newblock {\em arxiv/1912.12165}.

\bibitem{feng2016critical}
\textbf{Wenfeng Feng} and Richard Bailey.
\newblock Unifying relationships between complexity and stability in mutualistic ecological communities.
\newblock {\em Journal of Theoretical Biology}, 2018, 439: 100-126.

\bibitem{feng2014heterogeneity}
\textbf{Wenfeng Feng} and Kazuhiro Takemoto.
\newblock Heterogeneity in ecological mutualistic networks dominantly
  determines community stability.
\newblock {\em Scientific Reports}, 4:5912, August 2014.

\bibitem{feng2014climatic}
Kazuhiro Takemoto, Saori Kanamaru, and \textbf{Wenfeng Feng}.
\newblock Climatic seasonality may affect ecological network structure: Food
  webs and mutualistic networks.
\newblock {\em Biosystems}, 121:29--37, July 2014.

\bibitem{feng2013application}
\textbf{Wenfeng Feng} and Wenjuan Zhu.
\newblock Application of genetic algorithm and neural network in construction
  cost estimate.
\newblock {\em Advanced Materials Research}, 756:3194--3198, 2013.

\bibitem{wenfeng2012effect}
\textbf{Wenfeng Feng}, Yang Li.
\newblock Effect of interaction strength on the evolution of cooperation.
\newblock {\em arXiv:1209.2612}, 2012.

\bibitem{feng2011application}
\textbf{Wenfeng Feng} and WenJuan Zhu.
\newblock The application of sofm fuzzy neural network in project cost
  estimate.
\newblock {\em Journal of Software}, 6(8):1452--1459, 2011.

\bibitem{feng2011individual}
\textbf{Wenfeng Feng}, Yang Li, and Yan Jia.
\newblock Individual behaviour in group formation.
\newblock In {\em 2nd International Conference on Artificial Intelligence, Management Science and Electronic Commerce (AIMSEC) }, 2146--2148. IEEE, 2011.

\bibitem{heli2010peer}
Heli Xu, \textbf{Wenfeng Feng}, Yongfeng Huang, and Xing Li.
\newblock A peer-to-peer protocol in hierarchical navigable small-world
  network.
\newblock In {\em The 2nd International Conference on Computer and Automation Engineering (ICCAE)}, 5:519--522. IEEE, 2010.

\bibitem{feng2009gossip}
Zhibin Zhang, \textbf{Wenfeng Feng}, and Yongfeng Huang.
\newblock Gossip-based adaptive membership management protocol.
\newblock {\em Journal of Computer Applications}, 11:019, 2009. (in Chinese)

\bibitem{feng2008tsinghua}
\textbf{Wenfeng Feng}, Yongfeng Huang and Xing Li.
\newblock Reversible sketch data structure.
\newblock {\em Journal of Tsinghua University}, 48(10):1625--1628, 2008. (in Chinese)

\bibitem{liu2007peer}
Hui Liu, \textbf{Wenfeng Feng}, Yongfeng Huang, and Xing Li.
\newblock A peer-to-peer traffic identification method using machine learning.
\newblock In {\em Networking, International Conference on Architecture, and Storage. NAS'07.}, 155--160. IEEE, 2007.

\bibitem{feng2007hierarchical}
\textbf{Wenfeng Feng}, Qiao Guo, and Zhitao Guan.
\newblock Hierarchical count-min sketch data structure for data streams.
\newblock {\em Computer Engineering}, 33(14):20--23, 2007. (in Chinese)

\bibitem{feng2006network}
\textbf{Wenfeng Feng}, Qiao Guo, Li~Wang, and Fengcheng Liu.
\newblock Network traffic analysis system based on multidimensional data model.
\newblock {\em Computer Engineering}, 3:043, 2006. (in Chinese)

\bibitem{feng2006finding}
\textbf{Wenfeng Feng}, Qiao Guo, and Suyan Wu.
\newblock Finding frequent items of data streams based on hierarchical sketch.
\newblock {\em Transactions of Beijing Institute of Technology}, 6:011, 2006. (in Chinese)

\bibitem{feng2006xor}
\textbf{Wenfeng Feng}, Qiao Guo, and Zhibin Zhang.
\newblock A Xor-based hierarchical sketch for identifying and estimating hierarchical frequent items online.
\newblock In {\em First International Conference on Communications and Networking in China.}, 1--5. IEEE, 2006.

\bibitem{feng2006reversible}
\textbf{Wenfeng Feng}, and Zhibin Zhang.
\newblock Reversible sketch based on the xor-based hashing.
\newblock In {\em IEEE Asia-Pacific Conference on Services Computing. APSCC'06. }, 93--98. IEEE, 2006.

\end{thebibliography}

%\bibliographystyle{plain}
%\bibliography{mypapers}                        % 'publications' is the name of a BibTeX file

% Publications from a BibTeX file using the multibib package
%\section{Publications}
%\nocitebook{book1,book2}
%\bibliographystylebook{plain}
%\bibliographybook{publications}                   % 'publications' is the name of a BibTeX file
%\nocitemisc{misc1,misc2,misc3}
%\bibliographystylemisc{plain}
%\bibliographymisc{publications}                   % 'publications' is the name of a BibTeX file

\section{Skills}
\cvitem{Data analysis}{Proficient in R, Python-based deep learning frameworks such as Pytorch and Jax, Python-based data analysis tools such as NumPy, Pandas, Matplotlib, etc.}
\cvitem{Programming}{Proficient in Python and Python-based Web development framework such as Django.}
\cvitem{Knowledge management}{Conversant in using Zotero, Notes, etc. to manage, organize, track documents and create new ideas.}



%\clearpage\end{CJK*}                              % if you are typesetting your resume in Chinese using CJK; the \clearpage is required for fancyhdr to work correctly with CJK, though it kills the page numbering by making \lastpage undefined
\end{document}


%% end of file `template.tex'.
